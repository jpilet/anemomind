\documentclass{article}
\usepackage{textcomp}
\usepackage[english]{babel}
\usepackage[utf8]{inputenc}
\usepackage{lmodern}
\usepackage{textcomp}
\usepackage[T1]{fontenc}
\usepackage{ucs}
\usepackage{amssymb}
\usepackage{amsmath}
\usepackage{courier}
\usepackage{graphicx}
\usepackage{a4wide}

\newcommand{\quality}[1]{q_{#1}}
\newcommand{\minfun}{\text{min}}
\newcommand{\realvec}[1]{\mathbf{#1}}
\newcommand{\norm}[1]{\left\| #1 \right\|}
\newcommand{\derivative}[2]{\frac{\partial #1}{\partial #2}}
\newcommand{\timederivative}[1]{\derivative{#1}{t}}
\newcommand{\realnumber}{\mathbb{R}}
\setcounter{secnumdepth}{2}

\newcommand{\mat}[2]{\left( \begin{array}{#1} #2 \end{array} \right)}

\author{Jonas Östlund}
\date{\today}
\title{Linear calibration of a motion vector}

\begin{document}
\maketitle

A motion $M$ in the 2D plane can be expressed as an angle $\varphi$ and a magnitude $r$, so that the motion vector is
\begin{displaymath}
  M = r\mat{c}{
    \sin \phi \\
    \cos \phi
   } =
  \mat{c}{
    r \sin \phi \\
    r \cos \phi
    } \, .
\end{displaymath}
Scaling and rotation are linear operations and can be expressed as a matrix
\begin{displaymath}
\mat{cc}{
  s\cos \phi & -s \sin \phi \\
  s\sin \phi & s \cos \phi
} = \mat{cc}{a & -b \\ b & a} \qquad \text{with $a = s\cos \phi$ and $b = s\sin \phi$} \, .
\end{displaymath}
We might also want to correct for offset error $k$ in the velocity, thus the corrected motion $\hat{M}$ can be expressed as
\begin{displaymath}
\hat{M} = \mat{cc}{a & -b \\ b & a}\mat{c}{(r + k)\sin \varphi \\ (r + k)\cos \varphi} \, ,
\end{displaymath}
which can be rewritten as a data matrix multiplied by a vector with the parameters:
\begin{displaymath}
\hat{M} = \mat{cccc}{r\sin \varphi & -r \cos \varphi & 
  \sin \varphi & -\cos \varphi \\
  r\cos \varphi & r\sin \varphi & \cos \varphi & \sin \varphi}
\mat{c}{a \\ b \\ ak \\ bk} \, .
\end{displaymath}
Note that $\hat{M}$ is not a linear function of $(a, b, k)$. But if we relax the problem by introducing $c = ak$ and $d = bk$, it is linear in $(a, b, c, d)$. We can thus formulate linear least squares fitting problems.
\end{document}
